%% Definitions below taken from 4th year capstone Report template
\documentclass[12pt]{article}
\usepackage[margin=1in]{geometry}
\usepackage{amsmath,amsthm,amssymb}
\usepackage{hyperref}
\usepackage{float}    % Required to keep images where they are put
\usepackage{graphicx} % Required for the inclusion of images
\usepackage{listings} % Required for the inclusion of source code
\usepackage{mathtools}
\usepackage{enumitem} % Requred to use different characters for enumerated lists
\usepackage{multirow}
\usepackage[toc]{glossaries}
\usepackage[titletoc]{appendix}
\usepackage{wrapfig}
\usepackage{subfigure}
\setlength\parindent{0pt} % Removes all indentation from paragraphs
%\renewcommand{\labelenumi}{\alph{enumi}.} % Make numbering in the enumerate environment by letter rather than number (e.g. section 6)
\providecommand{\e}[1]{\ensuremath{\times 10^{#1}}} %use scientific notation
%\usepackage{times} % Uncomment to use the Times New Roman font
\usepackage[usenames,dvipsnames]{color}  % Required for colored syntax highlighting of code
\definecolor{listinggray}{gray}{0.9}
\definecolor{lbcolor}{rgb}{0.9,0.9,0.9}
\definecolor{DarkGreen}{rgb}{0.000000,0.392157,0.000000}
\lstset{
  backgroundcolor=\color{lbcolor},
  tabsize=4,
  language=c,
  captionpos=b,
  tabsize=3,
  frame=lines,
  numbers=left,
  numberstyle=\tiny,
  numbersep=5pt,
  breaklines=true,
  showstringspaces=false,
  basicstyle=\footnotesize,
%  identifierstyle=\color{magenta},
  keywordstyle=\color[rgb]{0,0,1},
  commentstyle=\color{DarkGreen},
  stringstyle=\color{red}
  }
\usepackage{pdfpages}
%\loadglsentries[main]{glossary}
%\makeglossaries
\usepackage[backend=bibtex, firstinits=false, style=ieee]{biblatex}
\addbibresource{Bibliog.bib}
\usepackage{verbatim}
%%Definitions Below are from 'Weekly Homework 3 template'
\newcommand{\N}{\mathbb{N}}
\newcommand{\R}{\mathbb{R}}
\newcommand{\Z}{\mathbb{Z}}
\newcommand{\Q}{\mathbb{Q}}
\newenvironment{theorem}[2][Theorem]{\begin{trivlist}
\item[\hskip \labelsep {\bfseries #1}\hskip \labelsep {\bfseries #2.}]}{\end{trivlist}}
\newenvironment{lemma}[2][Lemma]{\begin{trivlist}
\item[\hskip \labelsep {\bfseries #1}\hskip \labelsep {\bfseries #2.}]}{\end{trivlist}}
\newenvironment{exercise}[2][Exercise]{\begin{trivlist}
\item[\hskip \labelsep {\bfseries #1}\hskip \labelsep {\bfseries #2.}]}{\end{trivlist}}
\newenvironment{problem}[2][Problem]{\begin{trivlist}
\item[\hskip \labelsep {\bfseries #1}\hskip \labelsep {\bfseries #2.}]}{\end{trivlist}}
\newenvironment{question}[2][Question]{\begin{trivlist}
\item[\hskip \labelsep {\bfseries #1}\hskip \labelsep {\bfseries #2.}]}{\end{trivlist}}
\newenvironment{corollary}[2][Corollary]{\begin{trivlist}
\item[\hskip \labelsep {\bfseries #1}\hskip \labelsep {\bfseries #2.}]}{\end{trivlist}}
%%%%%%%%%%%%%%%%%%%%%%%%%%%%%%%%%%%%%%%%%%%%%%%%%%%%%%%%%%%%%%%%%%%%%%%%%%%%%%%%%%%%%%%%%%

\begin{document}

\title{Assignment 2}
\author{Pouyan Keshavarzian\\
ENEL 671: Adaptive Signal Processing}
\maketitle

\begin{problem}{1}
\[
  R=
    \begin{bmatrix}
      2 & 1 & 0.75 & 0.5 & 0.25 \\
      1 & 2 & 1 & 0.75 & 0.5 \\
      0.75 & 1 & 2 & 1 & 0.75 \\
      0.5 & 0.75 & 1 & 2 & 1 \\
      0.25 & 0.5 & 0.75 & 1 & 2
    \end{bmatrix}
\]
\begin{table}[H]
\centering
 \begin{tabular}{ | c | c | c |}
    \hline
    Filter Order & Eigenvalue Spread & Corresponding Upper Bound, $\mu$\\
    \hline\hline
    2        & 3.0 & 0.5 \\
    \hline
    3        & 4.2088 & 0.333 \\
    \hline
    4        & 5.0396 & 0.25\\
    \hline
    5        & 5.6864 & 0.2\\
    \hline
  \end{tabular}
  \caption{Problem 1 Calculations}
  \label{table:calcs}
\end{table}
You could not use a value close to the upperbound of the second order
filter for the fifth order because it exceeds the upper bound
therefore the filter would diverge.\\
\end{problem}

\begin{problem}{2}
\text{ }
\[
P=
  \begin{bmatrix}
    0.5 \\
    0.25 \\
    0.125\\
    0.0625\\
    0.03125
  \end{bmatrix}
\]\\
\end{problem}
\text{The calculated tap-input vectors for their corresponding filter orders are shown below: }\\
\[
W_02=
    \begin{bmatrix}
    0.25\\
    0\\
    \end{bmatrix}%\hspace*{20pt} % to increase the horizontal space
%
W_03=
    \begin{bmatrix}
    0.2571\\
    0.0.0179\\
    -0.0420\\
    \end{bmatrix}
W_04=
    \begin{bmatrix}
    0.2575\\
    0.0219\\
    -0.0325\\
    -0.0251\\
    \end{bmatrix}
W_05=
    \begin{bmatrix}
    0.2577\\
    0.0217\\
    -0.0331\\
    -0.0266\\
    0.0037\\
    \end{bmatrix}
\]
\begin{table}[H]
\centering
 \begin{tabular}{ | c | c | c |}
    \hline
    Filter Order & MMSE \\
    \hline\hline
    2        & 0.8750 \\
    \hline
    3        & 0.8723 \\
    \hline
    4        & 0.8714 \\
    \hline
    5        & 0.8714 \\
    \hline
  \end{tabular}
  \caption{Problem 2 Calculations}
  \label{table:calcs}
\end{table}
Filter order 4 and 5 give almost an identical MMSE as the filter has converged.\\
\end{document}
